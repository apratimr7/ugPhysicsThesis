
% \section{Introduction}
Nonlinearity is what makes world possible. Nonlinearity can bring about the conditions into the existence\ 
that would otherwise be impossible. If we look around every fascinating object in the world around has an\
element of depth and chaos. It is this intriguing quality of the universe that captivates our attention and\
keep the world from being boring. From the interaction of fundamental forces of nature, complex organism's biology\
to morden day digital technology and artificial intelligence, nonlinearity is at play \cite{feldmanChaosFractalsElementary2012,lakshmananNonlinearDynamics2003,strogatzNonlinearDynamicsChaos2018}. The central topic of this thesis\
is about one of the many captivating phenomenon of the nonlinearity in the world that has many potential application from fundamental physics to\
advancemenets in modern technology \cite{ackemannFundamentalsApplicationsSpatial}. Our discussion here will be focused on cavity solitons which are special type of solitary\
waves of light \cite{lugiatoIntroductionFeatureSection2003,ackemannFundamentalsApplicationsSpatial}. 

    Soliton was discovered accidently, when John Scott Russell (1844) was taking a walk by the Union Canal near Edinburg, saw a\
solitary wave travelling in the canal \cite{russellReportWavesMade}. Just take a moment and from a navie eye think about the phenomenon\
for a moment. It is extremely fascinating and weird. We usually get so captivated thinking about particles\ 
acting as waves from our quantum class, that we completely overlook to ask the question the other way around.\
Can a wave behave like a particle? There is a reason we never encounter any clue to the answer of this question\
in our usual physics classes. This is because the physics encountered in the classrooms are linear mostly while\
answer to these kind of questions lie in the more rich landscape of nonlinearity in nature.

One of the most interesting phenomena due to\
nonlinearity in nature, both for its applications and sheer mathematical beauty,\
is stable localized structures. Solitons are one such category of localized phenomena.\
Further, the cases of dissipative systems, which are usually far from equilibrium and close\
to the real world, also show the formation of Solitons. These kinds of solitons in the optical\
domain are broadly classified as Cavity Solitons (CS). They were first predicted in the early 1990s \cite{newellNonlinearOptics1992,firthTheoryCavitySolitons2001}\
to exist in nonlinear optical resonators as localized structures of light that are maintained due\
to a very intricate balance between diffraction or dispersion, nonlinearity, and feedback. Since their\
existence was experimentally confirmed in semiconductor microresonators, they have found various applications\
in different domains, including optical information technologies, photonics, and optical sensing. 

Trapping these cavity solitons is an important area of research. This is because trapping a CS\
gives us the ability to control and manipulate these localized light pulses within\
an optical cavity for application in advancements in technology. The real potential of the CS's entrapment is realised\
in photonics, where technologies such as optical switches can be used to make logic gates and effectively achieve optical\
computing. These optical switches are many orders of magnitude faster than regular semiconductor switches. 

\section{Background}
In optics, cavity solitons (CS) are beams of light in which the nonlinearity gets counter-balanced by diffraction 
and gain gets counter-balanced by loss in the system to get a stable structure that maintains it shape and form while 
propagating in the nonlinear medium. They were first theorized in 1980s by Moloney and colleagues \cite{newellNonlinearOptics1992} while 
studying transverse effects in optical bistability (OB). These are generated by using laser pulses in an optical cavity that contains 
a nonlinear medium driven by a coherent beam (holding beam) \cite{barlandCavitySolitonsPixels2002}. The ability to switch cavity solitons 
on and off and to control their location and motion by applying laser pulses makes them interesting as potential 'pixels' for reconfigurable 
arrays or all-optical processing units \cite{barlandCavitySolitonsPixels2002}. Theoretical work on cavity solitons has stimulated a variety 
of experiments in macroscopic cavities and in systems with optical feedback \cite{barlandCavitySolitonsPixels2002}.
The clear experimental realization has been hindered by boundary-dependence of the resulting optical patterns—cavity solitons should be self-confined. 
However, recent studies have demonstrated the generation of cavity solitons in vertical cavity semiconductor microresonators that are electrically 
pumped above transparency but slightly below lasing threshold \cite{barlandCavitySolitonsPixels2002}. 

The numerical simulations have allowed for clear interpretations of the experimental results and have served as an effective tool for theoretical study.
In a study by Firth et al. (2002), the dynamics of two-dimensional Kerr cavity solitons were analyzed, and found to be absolutely stable over a substantial 
parameter domain, with regions of stable oscillation and of fivefold or sixfold azimuthal instability beyond the instability boundary \cite{firthDynamicalPropertiesTwodimensional2002}.
McSloy et al. (2002) McSloy et al. (2002) applied quasi-exact numerical techniques to the calculation of stationary one- and two-dimensional, bound multipeaked cavity soliton solutions of a model describing a saturable absorber in a driven 
optical cavity \cite{mcsloyComputationallyDeterminedExistence2002}. Bache et al. (2005) studied a broad-area vertical cavity surface emitting laser (VCSEL) with a saturable absorber and numerically showed the presence of cavity solitons in 
the system \cite{bacheCavitySolitonLaser2005a}. These solitons existed as solitary structures formed through a modulationally unstable homogeneous lasing state that coexisted with a background with zero intensity \cite{bacheCavitySolitonLaser2005a}.

Cavity solitons are intriguing localized intensity peaks that have been studied extensively in various systems, including semiconductor microcavities and fiber lasers \cite{genevetCavitySolitonLaser2008,barbayCavitySolitonsVCSEL2011,songRecentProgressStudy2019}. The 
ability to control and manipulate these solitons makes them promising candidates for applications in all-optical processing units and other optical information systems \cite{aghdamiTwodimensionalDiscreteCavity2012,ackemannFundamentalsApplicationsSpatial,eslamiAllOpticalLogic2012,ghadiAllopticalComputingCircuits2021,mcmahonPhysicsOpticalComputing2023,randComputingSolitons2009}.

The complex Ginzburg-Landau equation (CGLE), is a close perturbative cousin of non-linear Schödinger's equation. The CGLE, initially devised for studying superconductivity \cite{aransonWorldComplexGinzburgLandau2002}, has found its way in nonlinear optics as a governing 
equation for soliton formation in nonlinear medium due to its pattern forming capabilities. The introduction of frequency-dependent losses has been shown to stabilize dissipative solitons in a complex Ginzburg-Landau model with simple cubic or saturable nonlinearity \cite{firthCavitySolitonProperties2009}. 
Analytic solutions for these solitons have been found in one dimension, and robust solitons have been identified numerically in two dimensions \cite{firthCavitySolitonProperties2009}. The research on cavity solitons in VCSELs and other systems using CGLE models has provided valuable and exciting insights which 
when confirmed using experimental data provides us with a robust theoretical tool \cite{hachairCavitySolitonsBroad2004,ackemannRealizationCavitysolitonLaser2007,paulauDriftingInstabilitiesCavity2009,firthCavitySolitonProperties2009}

\section{Motivation}

Little work has been done in analyising the formation of cavity solitons in periodic potentials. During recent works from our group\
it was noticed that periodic potentials with central peak can entrap a CS in a very interesting manner. These CS gets, trapped on/near the central peak\
of the potential. This is particularly interesting because successful implementation would give us an impressive ability to use light pulses to make a\
memory storage system. Just by controlling the position of the peak of the potential, one would be able to control the CS.  

This and many other essential technological applications are motivating factors for our current work. The prospects of the work lies in the fact that it can\
we can have a very effective way to mediate the data transfer and storage using Cavity Soliton. The theoretical modeling is the first step in physical prototyping.\
And this work aims to achieve the initial.

\section{Objective}
What follows is the theoretical modelling of cavity soliton formation in a well used laser cavity system VCSEL (more on this later) which can be modelled with complex Ginzburg-Landau\
equation. We aim to achive the following objectives by the end:
\begin{enumerate}
    \item Study the existence and stability of cavity solitons in various potential with central peak.
    \item Effective modelling of such phenomena using approtiate numerical techniques. 
\end{enumerate} 



The thesis is structured as follows. Chapter \ref{chap:Methodology} discusses some basics, methodology behind the approach and the system we have defined. 
Here we also take a brief look at the analytical formulation of CGLE and why we need to resort to numerical method in the light 
of the kind of systems we are dealing with. Further, in chapter \ref{chap:Results and Discussions}, we start by talking about our simulation, the assumptions we
have taken, and finally mentions the results we have had. Finally, in chapter \ref{chap:Conclusion} we summarize the report and conclude.  