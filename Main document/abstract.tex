We investigate the soliton formation and its dynamics inside a semiconductor laser cavity under central peak potentials. The nonlinear cavity comprises of a ventricle cavity surface emitting laser, coupled with saturable absorber and frequency selective feedback. We consider the cavity system is of (2+1) dimension. We achieved cavity soliton, which move randomly in absence of any externally applied potential, for only some selective ranges of system parameters. Further, we trap some of the cavity solitons using four different  potentials. This trapping has important technological merits 
in the field of nonlinear optics, particularly in all-optical switching, all-optical logic gate and optical computing.
This theoretical investigation involves numerical solution of a complex Ginzburg-Landau equation (CGLE); the governing equation of the cavity field, by Split-step Fourier Method (SSFM). Further, the stability of the cavity solitons has been investigated by using Lyapunov stability criteria. 